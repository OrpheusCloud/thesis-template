\chapter{Einleitung}
\label{ch:intro}
Test \citep{dueck:trio}. Test

%
% Section: Motivation
%
\section{Motivation}
\label{sec:intro:motivation}
%\graffito{Note: Beispiel Notiz an der Seite}
Globalisierung - Ein Schlagwort das Heute jeder schon gehört hat. Durch die Globalisierung werden viele Faktoren maßgäblich verändert, wie zum Beispiel verstärkte internationale Zusammenarbeit. 
Das Ziel ist es die eigene Wirtschaft immer mehr zu stärken und an allen Stellen zu optimieren, um das größtmögliche Wachsstum zu erreichen. Die Umsetzung des Ziels treibt auch den technologischen Fortschritt immer weiter voran. So wurden neue Kommunikationswege gefunden, um sich noch besser zu vernetzen.
Es ist nun Beispielsweise möglich, mit seinem Smartphone eine Nachricht an jemanden zu versenden, der auf einem anderen Kontinent lebt. Der Empfänger erhält die Nachricht und kann diese losgelöst von der Zeitzone und der Verfügbarkeit seines Kommuniikationspartners beantworten. 
Das Beantworten der Nachricht erfolgt hierbei asynchron, da der Empfänger die Nachricht jeder Zeit beantworten kann, wenn er sich dazu bereit fühlt.


Die asynchrone Kommunikation wird nun immer mehr zum Fundament in allen Berreichen. Beispiele hierfür sind Social Media und Messenger wie "Instagram" und "Telegram". Aber auch in der IT macht sich die asynchrone Kommunikation immer mehr bemerkbar. 
Systeme und Architekturen werden immer größer und komplexer durch das zunehmende Angebot und Nachfrage, die die Globalsisierung mit sich bringt. Die Unternehmen versuchen sich dabei von der Konkurrenz abzuheben, indem sie dem Kunden einen besseren Service oder ein besonderes Portfolio bieten. 
Die asynchrone Datenübertragung bei komplexen IT-Systemen funktioniert hierbei analog wie bei einem Messenger. Viele Daten werden bei einer anderen Schnittstelle angefragt und zu einem späteren Zeitpunkt empfangen.


Es gilt hierbei die Übermittlung und den Empfang der Daten so effizient wie möglich zu gestalten.
Diese Optimierung führt zu einer höheren Kundenzufriedenheit und damit nachhaltig zu mehr Umsatz.
Es gibt viele verschiedene Wege die Daten asynchron zu Übertragen. Jede Übertragungsart hat besondere Vor- und Nachteile.


Im Rahmen meiner Thesis werde ich mich genau mit dieser Thematik befassen. Es gilt herauszufinden, wie eine solche asynchrone Datenübertragung bei einem komplexen System realisiert werden muss, um die beste Wirtschaftlichkeit zu erzielen.


%
% Section: Ziele
%
\section{Ziel der Arbeit}
\label{sec:intro:goal}
Ei choro aeterno antiopam mea, ut eos erant homero concludaturque. Albucius appellantur deterruisset id eam, vivendum partiendo dissentiet ei ius. Vis melius facilisis ea, sea id convenire referrentur, takimata adolescens ex duo. Ei harum argumentum per. Eam vidit exerci appetere ad, ut vel zzril intellegam interpretaris.

Errem omnium ea per, pro \ac{UML} congue populo ornatus cu, ex qui dicant nemore melius. No pri diam iriure euismod. Graecis eleifend appellantur quo id. Id corpora inimicus nam, facer nonummy ne pro, kasd repudiandae ei mei. Mea menandri mediocrem dissentiet cu, ex nominati imperdiet nec, sea odio duis vocent ei. Tempor everti appareat cu ius, ridens audiam an qui, aliquid admodum conceptam ne qui. Vis ea melius nostrum, mel alienum ac elit id nibh pretium pulvina euripidis eu.

Ei choro aeterno antiopam mea, labitur bonorum pri no. His no decore nemore graecis. In eos meis nominavi, liber soluta vim cu. Integer consectetur, mi congue feugiat rhoncus, ante libero consectetur eros, et interdum nulla velit non velit. Mauris pharetra venenatis porttitor. Suspendisse et risus at dui gravida hendrerit. Aenean auctor interdum sodales. Etiam tortor orci, scelerisque in gravida eu, varius a massa. Ut sem odio, commodo id pharetra eu, dictum vitae. 

%
% Section: Struktur der Arbeit
%
\section{Gliederung}
\label{sec:intro:structure}
Nulla fastidii ea ius, exerci suscipit instructior te nam, in ullum postulant quo. Congue quaestio philosophia his at, sea odio autem vulputate ex. Cu usu mucius iisque voluptua. Sit maiorum propriae at, ea cum \ac{API} primis intellegat. Hinc cotidieque reprehendunt eu nec. Autem timeam deleniti usu id, in nec nibh altera.