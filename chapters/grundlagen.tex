\chapter{Grundlagen}
\label{ch:grundlagen}
Im Grundlagen Kapitel wird ein vertiefender Einblick in die theoretischen Grundlagen gegeben, die für die Thesis essenziell sind. 
Die Grundlagen werden in zwei Themengebiete aufgegliedert. Das erste Themengebiet ist die asynchrone Datenübertragung mit deren Vorgehensmodell. 
Das zweite Themengebiet gibt eine genaue Definition der komplexen Systeme, die für die Ausarbeitung relevant sind. 

\section{Asynchrone Datenübertragung}
Asynchrone Datenübertragung oder auch bekannt als „Asynchrone Kommunikation“ bezeichnet einen Übertragungsvorgang, bei dem Daten asynchron zwischen Server und Client kommuniziert werden. \cite{abts:2019} \\

Bei der Asynchronen Datenübertragung baut der Client eine Verbindung zum Server auf, welcher die Anfrage des Clients entgegennimmt und die Verbindung anschließend beendet. \cite{abts:2019} 
Der Client übermittelt hierbei zusätzlich den Port, auf dem er das Ergebnis empfangen möchte. 

Der Server verarbeitet anschließend die Anfrage, baut eine Verbindung über den erhaltenen Port zu dem Client auf und sendet diesem die angefragten Daten zurück. \\

Das Verarbeiten der Daten ist hierbei nicht zeitspezifisch. Der Server bearbeitet die Anfrage zeitversetzt und sendet diese zurück, sobald der Vorgang abgeschlossen ist. \cite*{tremp:2021} \\

Der alternative Übertragungsvorgang ist die synchrone Datenübertragung oder auch „Synchrone Kommunikation“ genannt. Dabei wartet der Client auf die Antwort des Servers und verarbeitet diese anschließend direkt weiter. 

Im Rahmen dieser Ausarbeitung wird ausschließlich der asynchrone Datenübertragungsvorgang behandelt, da die Synchrone Datenübertragung nichts mit der Thematik zu tun hat und somit nicht zielführend für die Problemstellung dieser Ausarbeitung ist. \\

Es gibt mehrere verschiedene Methoden eine asynchrone Datenübertragung umzusetzen. Die verschiedenen Methoden und weitere Grundlagen werden in den folgenden Abschnitten dargestellt. 

\subsection{Methoden der Asynchonen Datenübertragung}
Nach aktuellen Trends der Informatik gibt es drei verschiedene Methoden für die Implementierung der asynchronen Datenübertragung. 

Bei den Methoden handelt es sich um Asynchrone Rückruffunktionen, Broker-basierte Asynchrone Datenübertragung sowie Polling/Websockets und werden in den folgenden Abschnitten beschrieben. 

\subsubsection{Asynchrone Rückruffunktionen}
Bei den Asynchronen Rückruffunktionen oder auch „Asynchrone Callbacks“ genannt, muss der Client sich zunächst bei dem Server registrieren, indem der Server eine Referenz auf den Client speichert. \cite{abts:2019}

Der Client ruft mittels Webservice oder \ac{RPC} die entfernte Server Methode auf. Sobald der Server die Ereignisbehandlungsroutinen abgearbeitet und die angeforderten Daten ermittelt hat, benachrichtigt dieser den Client. \cite*{bengelbaun:2015}

Die Umsetzung des RPCs erfolgt hierbei ebenfalls asynchron, sodass der Client sofort weiterarbeiten kann. \cite*{schill:2012}

Der Client kann nun die Daten anfordern oder sich diese über einen RPC vom Server schicken lassen. Der Server ruft dabei eine entfernte Methode des Clients auf. \cite*{abts:2019,schill:2012}  \\

Der Vorteil bei den Asynchronen Rückruffunktionen gegenüber der Polling-Methode für die asynchrone Datenübertragung liegt darin, dass die Performance der Rückruffunktionen besser ist. Dies liegt an der geringeren Rechenzeit und Netzlast. \cite*{abts:2019}

Die asynchronen Remote Procedure Calls erleichtern zusätzlich das Versenden großer Datenmengen in kurzer Zeit oder das Durchführen langer und komplexer Berechnungen. \cite*{schill:2012}

\subsubsection{Broker-basierte Asynchrone Datenübertragung}
Die Broker-basierte Asynchrone Datenübertragung funktioniert analog zu den Asynchronen Rückruffunktionen. Der Haupt-Unterschied zwischen den beiden Methoden liegt darin, dass bei der Broker-basierte Asynchrone Datenübertragung, wie der Name bereits impliziert, ein Broker zwischengeschaltet ist, der die Nachrichten verwaltet. 

Bei der Broker-basierten Asynchronen Datenübertragung gibt es mehrere verschiedene Modelle die asynchron umgesetzt werden können. Die Modelle umfassen das Point-to-Point-Modell, das Request/Reply-Modell und das Publish/Subscribe-Modell. \cite*{abts:2019,tremp:2021}

Im Rahmen dieser Ausarbeitung wird hier lediglich das Publish/Subscribe-Modell als Beispiel für eine Broker-basierte Asynchrone Datenübertragung aufgegriffen, da dieses Modell leichter asynchron umzusetzen und besser in das komplexe System integrierbar ist. \\

Die Broker-basierte Asynchrone Datenübertragung basiert auf einer \ac{MOM} um asynchrone Datenübertragung zu ermöglichen. Message Oriented Middelware besteht aus einem Broker und einem asynchronen Kommunikationsprotokoll. \cite*{tremp:2021}

Ein Broker ist zur Verwaltung verschiedener Warteschlangen, die mit Nachrichten befüllt sind, sowie dem Speichern und Vermitteln der Nachrichten. \cite*{tremp:2021}

Bei dem Publish/Subscribe-Modell werden Nachrichten von einem Sender zu einem Bestimmten Thema (topic) veröffentlicht (published). Der Client abonniert ein Thema bei dem MOM-Server. Der MOM-Server vermittelt anschließend die Nachricht zu einem Topic an alle Subscriber, die das Thema abonniert haben. \cite*{abts:2019} Publisher und Subscriber agieren hierbei unabhängig voneinander wodurch die Daten asynchron übertragen werden. Jeder Subscriber empfängt dieselbe Nachricht von einem abonnierten Thema. \cite*{tremp:2021}

Je nach Realisierung ist eine aktive Verbindung zwischen dem Subscriber und dem MOM-Server notwendig oder nicht notwendig, um nachträglich alle Nachrichten zu einem Thema zu erhalten. \cite*{abts:2019}

\subsubsection{Polling/Websockets}
Die dritte Methode, um asynchrone Datenübertragung umzusetzen ist Polling. Polling beschreibt das regelmäßige Absenden von Anfragen an einen Server, bis das gewünschte Ergebnis geliefert wird. \cite*{goll:2020} \\

Der Client registriert sich zunächst bei einem Server. Der Client sendet einen request an den Server, um die benötigten Daten zu erhalten. Der Server bestätigt anschließend die Anfrage und verarbeitet dann die Response. \cite*{goll:2020} Der Client kann nun weiterarbeiten und sendet nun parallel und frequentiell Anfragen an den Server, um den Status zu ermitteln, bis das Ergebnis (die Response des Servers) empfangen werden kann. Der Client hat nun die gewünschten Daten erhalten. Der Vorgang ist hierbei asynchron, da der Client weiterarbeitet bis die Daten eintreffen und nicht blockiert. \\

Je nach Kontext ist es sinnvoller Websockets zu implementieren. Websockets ermöglichen im Gegensatz zum Polling, dass bidirektionale Versenden von Nachrichten. Dem Server ist es damit ebenfalls möglich den Client Benachrichtigungen zu senden. Der Client kann dadurch gezielt eine Anfrage für die Ergebnisdaten schicken und muss diese nicht immer wieder wie beim Polling versenden. Websockets sind somit wesentlich performanter und haben eine geringere Netzlast. \cite*{goll:2020}

\subsection{Anforderungen und Vorteile einer Asynchronen Datenübertragung}
Die Anforderungen, die eine asynchrone Datenübertragung erfüllen muss, sind überschaubar. Die besonderen Anforderungen differenzieren sich im Vergleich zu synchroner Datenübertragung lediglich in dem Punkt, dass empfangene Nachrichten zeitverschoben verarbeitet werden \cite*{tremp:2021} und der Sender somit weitere Aufgaben bis zum Eintreffen des Ergebnisses umsetzen kann.  \\

Spezifische Vorteile der Asynchronen Datenübertragung gegenüber der synchronen Datenübertragung liegen darin, dass eine lose Kopplung zwischen den Applikationen vorliegt, der Ressourcenverbrauch gering ist, sowie die Fehlertoleranz hoch. \cite*{tremp:2021}
Weitere Vorteile liegen in der guten Skalierbarkeit und der schnellen, parallelen Verarbeitung. \cite*{tremp:2021}

Die hohe Fehlertoleranz wird durch die lose Kopplung von Sender und Empfänger gewährleistet. Die beteiligten Komponenten können unabhängig voneinander neugestartet und Nachrichten nachträglich empfangen werden. \cite*{abts:2019}

Nachrichten können an mehrere Empfänger gleichzeitig zugestellt werden bei Broker-basierte Asynchrone Datenübertragung, da der Broker die Nachrichten an jeden Subscriber übermittelt. \cite*{abts:2019}

Die Kommunikation funktioniert unabhängig von Programmiersprache und Plattform, durch Verwendung von nachrichtenorientierter Middleware. \cite{abts:2019} 

\section{Komplexe Systeme}
Hier Text ergänzen.


\subsection{Definition}
Ein komplexes System wird im Rahmen dieser Ausarbeitung wie folgt definiert:

Bei einem komplexen System handelt es sich um ein Verteiltes System komplexer Natur. Das Verteilte System umfasst mehrere Komponenten und ist in eine Systemlandschaft mit mindestens zehn Schnittstellen zu anderen Systemen eingebettet. Die Anzahl der Komponenten hat hierbei keine Begrenzung nach oben und muss eine Skalierbarkeit für n weitere Komponenten besitzen. Die mindeste Anzahl an Komponenten bezieht sich hierbei auf die geringstmögliche Anzahl an Komponenten, die für das Verteilte System benötigt werden und variiert je nach Systemarchitektur und Anforderungen an das System, die von dem jeweiligen Unternehmen definiert wird. Der Umfang an Komponenten, Schnittstellen und Systemabhängigkeiten muss jeweils von einem IT-Experten bewertet werden, der mit der Systemarchitektur vertraut ist. 

Die Verteilten Systeme werden aufgrund verschiedener IT-Trends komplexer. So muss ein Verteiltes System immer kürzere Entwicklungszeit vorweisen, um schneller eingesetzt werden zu können und dabei möglichst wenig Entwicklungskosten verursachen. \cite{vorossanchez:2003}

\subsection{Verteiltes System}

\subsubsection{Client/Server-Modell}
Bei dem Client-Server-Modell handelt es sich um einen traditionellen Ansatz \cite*{schill:2012} der am weitesten verbreitet ist für die verteilte Verarbeitung. \cite*{abts:2019}

Bei dem Modell gibt es mehrere Clients, die einem Server Anfragen schicken, um Services durchzuführen oder Daten zu erhalten. Der Server stellt dabei die Dienste oder Daten dem Client zur Verfügung und kann mehrere Clients gleichzeitig bedienen. \cite*{bengelbaun:2015}
Der Server kann sich hierbei falls benötigt, bei Services anderer Server bedienen, um die Anfrage des Clients abzuarbeiten. \cite{schill:2012}

Client und Server können auf dem gleichen Rechner oder auf unterschiedlichen Rechnern ausgeführt werden. \cite{bengelbaun:2015}
Diese werden in der Praxis meist auf unterschiedlichen Rechnern verteilt, um die Vorteile eines Verteilten Systems nutzen zu können. \cite{abts:2019}

Die Kommunikation zwischen Client und Server wird meist mit Remote Procedure Calls umgesetzt, um ferne Prozeduraufrufe durchführen zu können. \cite*{schill:2012}

Die Anfragen des Clients an den Server erfolgen meist synchron, können jedoch ebenfalls asynchron umgesetzt werden. \cite{schill:2012}
Der Client kann hierbei weitere Anfragen an den Server senden, noch bevor der Server die vorherigen Aufgaben beendet hat. Die Anfragen werden dabei in separaten Threads abgearbeitet. \cite{abts:2019}

\subsubsection{Komponentenmodell}

\subsection{Schnittstellen}

\subsection{Eigenschaften}

\section{Wirtschaftlichkeit}
