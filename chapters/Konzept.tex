\chapter{Konzept}
\label{ch:Konzept}
In diesem Kapitel wird das genaue Vorgehen für die Evaluation der asynchronen Datenübertragungswege dargestellt. Zunächst werden die Rahmenbedingungen und Voraussetzungen für die Evaluation offengelegt. Anschließend erfolgt eine Planung für das Vorgehen, wie genau im nächsten Kapitel evaluiert werden soll und wie das Ergebnis getestet werden kann. Der letzte Abschnitt umfasst die Annahme des Ergebnisses, die vor der praktischen Durchführung, basierend auf der Forschungsfrage, angestellt werden.

\section{Rahmenbedingungen und Voraussetzungen}
Um die Evaluation korrekt durchzuführen, müssen einige Rahmenbedingungen und Voraussetzungen gegeben sein. 

Bei dem zugrunde liegenden System muss es sich um ein komplexes System handeln. Die dafür notwendigen Mindest-Anforderungen müssen alle erfüllt sein.  

Außerdem muss das System die Anforderung haben Daten asynchron zu übertragen. Es gibt Fälle, in denen eine asynchrone Datenübertragung keinen sinnvollen Nutzen mit sich bringt oder eine synchrone Datenübertragung zwangsläufig notwendig ist.

Des Weiteren müssen alle Methoden der asynchronen Datenübertragung in das System integrierbar sein. Andernfalls lassen sich die Methodiken nicht vergleichen und die Evaluation ist nur bedingt übertragbar. So müssen Unternehmens-Interne Aspekte betrachtet werden, da Firewalls, Security, Compliance und Lizenzen eine Rolle spielen können und somit einzelne Methodiken der asynchronen Datenübertragung kategorisch ausschließen. Es gilt zusätzlich zu überprüfen ob weitere Berechtigungen oder Schnittstellen notwendig sind, um die asynchrone Datenübertragungsmethodiken umsetzen zu können. 

Zusätzlich gilt, dass die Gewichtungen der Evaluationskriterien durch eigene IT-Experten vorgenommen werden muss, sofern die Priorisierungen abweichen. So können Kriterien wie die Umsetzbarkeit stark von dem Budget des Unternehmens abhängen. Ein Großunternehmen weist meist wesentlich mehr Budget auf als ein kleines oder ein mittelständiges Unternehmen, jedoch können alle die Anforderungen eines komplexen Systems erfüllen. 

Für die Evaluationskriterien und Bewertung wird zusätzlich vorausgesetzt, dass die IT-Experten, die die Bewertung und Priorisierung vornehmen, sich mit der Thematik der Datenübertragung auskennen, sowie gute Kenntnisse über die Architektur ihres vorliegenden Komplexen Systems haben.

\section{Planung der Evaluation}