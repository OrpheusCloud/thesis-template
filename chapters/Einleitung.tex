\chapter{Einleitung}
\label{ch:intro}
%
% Section: Motivation
%
\section{Motivation}
\label{sec:intro:motivation}
%\graffito{Note: Beispiel Notiz an der Seite}
Globalisierung - Ein Schlagwort das Heute jeder schon gehört hat. Durch die Globalisierung werden viele Faktoren maßgäblich verändert, wie zum Beispiel verstärkte internationale Zusammenarbeit. 
Das Ziel ist es die eigene Wirtschaft immer mehr zu stärken und an allen Stellen zu optimieren, um das größtmögliche Wachsstum zu erreichen. Die Umsetzung des Ziels treibt auch den technologischen Fortschritt immer weiter voran. So wurden neue Kommunikationswege gefunden, um sich noch besser zu vernetzen.
Es ist nun Beispielsweise möglich, mit seinem Smartphone eine Nachricht an jemanden zu versenden, der auf einem anderen Kontinent lebt. Der Empfänger erhält die Nachricht und kann diese losgelöst von der Zeitzone und der Verfügbarkeit seines Kommuniikationspartners beantworten. 
Das Beantworten der Nachricht erfolgt hierbei asynchron, da der Empfänger die Nachricht jeder Zeit beantworten kann, wenn er sich dazu bereit fühlt. \\

Die asynchrone Kommunikation wird nun immer mehr zum Fundament in allen Berreichen. Beispiele hierfür sind Social Media und Messenger wie "Instagram" und "Telegram". Aber auch in der IT macht sich die asynchrone Kommunikation immer mehr bemerkbar. 
Systeme und Architekturen werden immer größer und komplexer durch das zunehmende Angebot und Nachfrage, die die Globalsisierung mit sich bringt. Die Unternehmen versuchen sich dabei von der Konkurrenz abzuheben, indem sie dem Kunden einen besseren Service oder ein besonderes Portfolio bieten. 
Die asynchrone Datenübertragung bei komplexen IT-Systemen funktioniert hierbei analog wie bei einem Messenger. Viele Daten werden bei einer anderen Schnittstelle angefragt und zu einem späteren Zeitpunkt empfangen.\\

Es gilt hierbei die Übermittlung und den Empfang der Daten so effizient wie möglich zu gestalten.
Diese Optimierung führt zu einer höheren Kundenzufriedenheit und damit nachhaltig zu mehr Umsatz.
Es gibt viele verschiedene Wege die Daten asynchron zu Übertragen. Jede Übertragungsart hat besondere Vor- und Nachteile.\\

Im Rahmen meiner Thesis werde ich mich genau mit dieser Thematik befassen. Es gilt herauszufinden, wie eine solche asynchrone Datenübertragung bei einem komplexen System realisiert werden muss, um die beste Wirtschaftlichkeit zu erzielen.


%
% Section: Ziele
%
\section{Ziel der Arbeit}
\label{sec:intro:goal}
Das Hauptziel der Arbeit ist es, zu ermitteln, was die sinnvollste asynchrone Datenübertragungsart für ein komplexes System eines Großunternehmens ist. 

Des Weiteren soll ermittelt werden was es für asynchrone Datenübertragungsarten es gibt, was deren Vor- und Nachteile sind und wie sich diese implementieren lassen. 

Ein weiteres Ziel ist die Evaluation der Übertragungsarten hinsichtlich besonderer Kriterien. 

Der Leser soll mein Ergebnis auf sein komplexes System übertragen und die sinnvollste asynchrone Datenübertragungsart anhand meiner Evaluation bestimmen können. 


%
% Section: Struktur der Arbeit
%
\section{Struktur der Arbeit}
\label{sec:intro:structure}
Zunächst werden die relevanten Grundlagen behandelt, welche für die weitere Thematik der Arbeit relevant sind. 
Diese unterteilen sich in die beiden Bereiche asynchrone Datenübertragung und dem Bereich komplexe Systeme. \\

Im Bereich der asynchronen Datenübertragung wird die Basis für den späteren Kern der Ausarbeitung geschaffen. 
Dabei wird dargestellt, was für Vorgehensmodelle es bei der asynchronen Datenübertragung gibt und wie diese realisiert werden.\\

Der Abschnitt komplexe Systeme beschreibt genauer um was für Systeme es sich handelt, was diese charakterisiert und wie diese aufgebaut sind.\\

Im Hauptteil der Arbeit werden die Rahmenbedingungen, Voraussetzungen und die Kriterien genauer dargestellt und wie die Kriterien zustande kommen, um die asynchronen Übertragungswege zu evaluieren. 
Anschließend wird genauer dargestellt, wie die Evaluation abläuft, wie man diese zusätzlich testen kann und was die konkreten Annahmen für die Realisierungen sind. \\

Im Kern der Ausarbeitung wird die Evaluation konkret realisiert und eine kleine Demo zu den Vorgehens Modellen dargestellt. \\

Zum Schluss werden die wichtigsten zuvor ermittelten Aspekte zusammengefasst und das Ergebnis ausdiskutiert, inwieweit das Ergebnis den Annahmen entsprach und wo die Grenzen des Resultats liegen. 
Zusätzlich wird ein Ausblick gegeben, inwieweit man das Ergebnis der Ausarbeitung auf andere Probleme der Informatik übertragen kann und wie nachhaltig relevant die Ausarbeitung sein wird.